\section{Introduction}
Include any set up for the experiment. This could be as follows:
\begin{itemize}
\item Architecture:          x86\_64
\item CPU op-mode(s):        32-bit, 64-bit
\item Byte Order:            Little Endian
\item CPU(s):                4
\item On-line CPU(s) list:   0-3
\item Thread(s) per core:    2
\item Core(s) per socket:    2
\item Vendor ID:             GenuineIntel
\item CPU family:            6
\item Model:                 61
\item CPU MHz:               500.000
\item RAM:					 8Gb
\end{itemize}
\section{Experiment 1}
Use graphics to display results. Most results can be shown in tables. Use package longtables if you have trouble getting all data into a single page. Remember to label and reference the table.

\begin{table}
	\centering
	\begin{tabular}{|c|l|l|l||r|}\hline
	Run	&	x	&	y 	&	z	& Time/ms\\ \hline
	1	&	100	&	2.5	& 	1.06	& 60000 \\ \hline
	2	&	200	&	4.5	&	1.45	& 50000 \\ \hline
	3	&	500	&	5.5	&	1.5	& 40000 \\ \hline
	\end{tabular}
	\caption{Experiment 1 Results}
	\label{ta:ex1}
\end{table}


\section{Experiment 2}
Experiment 2 should differ from Experiment 1. Each experiment should be repeated a number of times for reproducibility. Do not confuse Experiment 2 as a repeat of Experiment 1, they are different. 

\begin{table}
	\centering
	\begin{tabular}{|c|l|l|l||r|}\hline
	Run	&	x	&	y 	&	z	& Time/ms\\ \hline
	1	&	100	&	2.5	& 	1.06	& 60000 \\ \hline
	2	&	200	&	4.5	&	1.45	& 50000 \\ \hline
	3	&	500	&	5.5	&	1.5	& 40000 \\ \hline
	\end{tabular}
	\caption{Experiment 2 Results}
	\label{ta:ex2}
\end{table}




\section{Summary}
Bad results happen, but it is not bad science. Still write up the results. If given time run more experiments. Rejecting or accepting an hypothesis is still a result worth writing up.

Good results will yield a clear direction and give clear recommendations and guidelines that can be mentioned here and emphasised in the next chapter \-- it is OK to repeat.


